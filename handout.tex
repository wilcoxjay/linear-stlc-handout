\documentclass{article}
\usepackage[margin=1in]{geometry}
\usepackage{amsmath}
\usepackage{amsthm}
\usepackage{amssymb}
\usepackage{stmaryrd}
\usepackage{tikz}
\usepackage{mathpartir}
\usepackage{array}
\usepackage{varwidth}
\usepackage{xcolor}

\title{\vspace{-60pt}A Missing 505 Lecture: Linear STLC}
\author{Chandrakana Nandi\and James R. Wilcox}
\date{January 26, 2018}

\newcommand{\bnfeq}{&::=&}
\newcommand{\bnfalt}{&~\mid&}

\newcommand{\todo}[1]{{\color{red} #1}}

\newcommand{\kw}[1]{\ensuremath{\mathsf{#1}}}

\newcommand{\unit}{\ensuremath{\kw{unit}}}
\newcommand{\bool}{\ensuremath{\kw{bool}}}
\newcommand{\tyint}{\ensuremath{\kw{int}}}
\newcommand{\tyarray}{\ensuremath{\kw{array}}}
\newcommand{\linto}{\ensuremath{\multimap}}

\newcommand{\true}{\ensuremath{\kw{true}}}
\newcommand{\false}{\ensuremath{\kw{false}}}
\newcommand{\IF}[3]{\ensuremath{\kw{if}\ #1\ \kw{then}\ #2\ \kw{else}\ #3}}
\newcommand{\gop}{\ensuremath{\circ}}

\newcommand{\letin}[2]{\kw{let}\ #1 = #2\ \kw{in}}

\newcommand{\Colon}{{\hspace{1pt}:\hspace{1pt}}}
\newcommand{\COLON}{\mathrel{\,:\,}}

\newcommand{\LAMBDA}[3]{\lambda #1\Colon#2.\; #3}
\newcommand{\fun}{\ensuremath{\kw{fun}}}
\newcommand{\fix}[5]{\fun\ #1\ (#2\Colon#3):#4 = #5}

\newcommand{\hastype}[3]{\ensuremath{#1\mathrel{\,\vdash\,}#2 \COLON #3}}
\newcommand{\haslintype}[3]{\ensuremath{#1\mathrel{\,\Vdash\,}#2 \COLON #3}}

\newcommand{\step}[2]{\ensuremath{#1~\longrightarrow~#2}}

\begin{document}
\maketitle

\section*{But first: Structural CBV STLC + recursion, bools, and ints}

{
\noindent
\hskip 1cm plus 1fill minus 0pt
\begin{varwidth}[t]{0.5\textwidth}
\[
\begin{array}{r@{~}c@{~~}l}
  \tau \bnfeq
       %\unit\\
       %\bnfalt
       \bool\\
       \bnfalt \tyint\\
       \bnfalt \tau\to\tau\\\\
  \oplus\bnfeq
       +\\
       \bnfalt -\\
       \bnfalt =\\
       \bnfalt\ldots\\
\end{array}
\]
\end{varwidth}
\hskip 0pt plus 1fill minus 0pt
\begin{varwidth}[t]{0.5\textwidth}
\[
\begin{array}{r@{~}c@{~~}l}
  e \bnfeq     x \\
    %\bnfalt    \LAMBDA{x}{\tau}{e}\\
    \bnfalt    \fix{f}{x}{\tau_1}{\tau_2}{e}\\
    \bnfalt    e\ e \\
    %\bnfalt    () \\
    \bnfalt    \true \\
    \bnfalt    \false \\
    \bnfalt    \IF{e}{e}{e}\\
    \bnfalt    n \\
    \bnfalt    e \oplus e \\

\end{array}
\]
\end{varwidth}
\hskip -1cm plus 1fill minus 0pt
\begin{varwidth}[t]{0.5\textwidth}
\[
\begin{array}{r@{~}c@{~~}l}
  v \bnfeq     \fix{f}{x}{\tau_1}{\tau_2}{e}\\
    %\bnfalt    \LAMBDA{x}{\tau}{e}\\
    %\bnfalt    () \\
    \bnfalt    \true \\
    \bnfalt    \false\\
    \bnfalt    n\\
\end{array}
\]
\end{varwidth}
\hskip 0pt plus 1fill minus 0pt
}

\vspace{5mm}

{
\def\MathparLineskip {\lineskiplimit=1.2em\lineskip=2.5em plus 0.2em}

{
\setlength{\fboxsep}{5pt}
\fbox{\hastype{\Gamma}{e}{\tau}}
}
\begin{mathpar}
  \inferrule*[right=Var]
    {x\Colon\tau\in\Gamma}
    {\hastype{\Gamma}{x}{\tau}}
  \and
  % \inferrule*[right=Abs]
  %   {\hastype{\Gamma,x\Colon\tau_1}{e}{\tau_2}}
  %   {\hastype{\Gamma}{\LAMBDA{x}{\tau_1}{e}}{\tau_1\to\tau_2}}
  \inferrule*[right=Fun]
    {\hastype{\Gamma,f\Colon\tau_1\to\tau_2,x\Colon\tau_1}{e}{\tau_2}}
    {\hastype{\Gamma}{\fix{f}{x}{\tau_1}{\tau_2}{e}}{\tau_1\to\tau_2}}
  \and
  \inferrule*[right=App]
    {\hastype{\Gamma}{e_1}{\tau_1\to\tau_2} \and
     \hastype{\Gamma}{e_2}{\tau_1}}
    {\hastype{\Gamma}{e_1\ e_2}{\tau_2}}
  % \and
  % \inferrule*[right=Unit]
  %   { }
  %   {\hastype{\Gamma}{()}{\unit}}
  \and
  \inferrule*[right=True]
    { }
    {\hastype{\Gamma}{\true}{\bool}}
  \and
  \inferrule*[right=False]
    { }
    {\hastype{\Gamma}{\false}{\bool}}
  \and
  \inferrule*[right=Int]
    { }
    {\hastype{\Gamma}{n}{\tyint}}
  \and
  \inferrule*[right=If]
    {\hastype{\Gamma}{e_1}{\bool}\and
     \hastype{\Gamma}{e_2}{\tau}\and
     \hastype{\Gamma}{e_3}{\tau}}
    {\hastype{\Gamma}{\IF{e_1}{e_2}{e_3}}{\tau}}
  \and
  \inferrule*[right=Op]
    {\oplus\COLON \tau_1^2 \to \tau_2 \and
     \hastype{\Gamma}{e_1}{\tau_1} \and
     \hastype{\Gamma}{e_2}{\tau_1}
    }
    {\hastype{\Gamma}{e_1\oplus e_2}{\tau_2}}
\end{mathpar}
}

{
\def\MathparLineskip {\lineskiplimit=1.2em\lineskip=2.5em plus 0.2em}

{
\setlength{\fboxsep}{5pt}
\fbox{\step{e}{e}}
}
\begin{mathpar}
  % \inferrule*[right=Beta]
  %   { }
  %   {\step{(\LAMBDA{x}{\tau}{e_1})\ v_2}{e_1[v_2/x]}}
  \inferrule*[right=Beta]
    { v_1 ~=~~ \fix{f}{x}{\tau_1}{\tau_2}{e_1}}
    {\step{v_1\ v_2}{e_1[v_1,v_2/f,x]}}
  \and
  \inferrule*[right=App1]
    {\step{e_1}{e_1'}}
    {\step{e_1\ e_2}{e_1'\ e_2}}
  \and
  \inferrule*[right=App2]
    {\step{e_2}{e_2'}}
    {\step{v_1\ e_2}{v_1\ e_2'}}
  \and
  \inferrule*[right=IfT]
    { }
    {\step{\IF{\true}{e_2}{e_3}}{e_2}}
  \and
  \inferrule*[right=IfF]
    { }
    {\step{\IF{\false}{e_2}{e_3}}{e_3}}
  \and
  \inferrule*[right=IfInd]
    {\step{e_1}{e_1'}}
    {\step{\IF{e_1}{e_2}{e_3}}{\IF{e_1'}{e_2}{e_3}}}
  \and
  \inferrule*[right=Op1]
    {\step{e_1}{e_1'}}
    {\step{e_1\oplus e_2}{e_1'\oplus e_2}}
  \and
  \inferrule*[right=Op2]
    {\step{e_2}{e_2'}}
    {\step{v_1\oplus e_2}{v_1\oplus e_2'}}
  \and
  \inferrule*[right=DoOp]
    { \llbracket\oplus\rrbracket(v_1,v_2)=v_3}
    {\step{v_1\oplus v_2}{v_3}}
\end{mathpar}
}

\clearpage

\section*{Adding purely functional arrays}

We can then add purely functional arrays, where writing copies the
whole thing. (This is actually kind of garbage since programs that
access out of bounds get stuck. So these are unsafe immutable arrays.)

{
\noindent
\hskip 1cm plus 1fill minus 0pt
\begin{varwidth}[t]{0.5\textwidth}
\[
\begin{array}{r@{~}c@{~~}l}
  \tau \bnfeq \ldots\\
       \bnfalt \tyarray\ \tau\\
\end{array}
\]
\end{varwidth}
\hskip 0pt plus 1fill minus 0pt
\begin{varwidth}[t]{0.5\textwidth}
\[
\begin{array}{r@{~}c@{~~}l}
  e \bnfeq     \ldots\\
    \bnfalt    [e, \ldots, e]\\
    \bnfalt    e[e]\\
    \bnfalt    e[e] := e\\
\end{array}
\]
\end{varwidth}
\hskip -1cm plus 1fill minus 0pt
\begin{varwidth}[t]{0.5\textwidth}
\[
\begin{array}{r@{~}c@{~~}l}
  v \bnfeq     \ldots\\
    \bnfalt    [v, \ldots, v]\\
\end{array}
\]
\end{varwidth}
\hskip 0pt plus 1fill minus 0pt
}

{
\def\MathparLineskip {\lineskiplimit=1.2em\lineskip=2.5em plus 0.2em}

{
\setlength{\fboxsep}{5pt}
\fbox{\hastype{\Gamma}{e}{\tau}}
}
\begin{mathpar}
  \inferrule*[right=Alloc]
    {\forall i,~\hastype{\Gamma}{e_i}{\tau}}
    {\hastype{\Gamma}{[e_1,\dots,e_n]}{\tyarray\ \tau}}
  \and
  \inferrule*[right=Read]
    {\hastype{\Gamma}{e_1}{\tyarray\ \tau} \and
     \hastype{\Gamma}{e_2}{\tyint}
    }
    {\hastype{\Gamma}{e_1[e_2]}{\tau}}
  \and
  \inferrule*[right=Write]
    {\hastype{\Gamma}{e_1}{\tyarray\ \tau} \and
     \hastype{\Gamma}{e_2}{\tyint} \and
     \hastype{\Gamma}{e_3}{\tau}
    }
    {\hastype{\Gamma}{e_1[e_2] := e_3}{\tyarray\ \tau}}
\end{mathpar}
}

{
\def\MathparLineskip {\lineskiplimit=1.2em\lineskip=2.5em plus 0.2em}

{
\setlength{\fboxsep}{5pt}
\fbox{\step{e}{e}}
}
\begin{mathpar}
  \inferrule*[right=AllocInd]
    {\step{e_i}{e_i'}}
    {\step{[v_1,\dots,v_{i-1},e_i,\dots,e_n]}{[v_1,\dots,v_{i-1},e_i',\dots,e_n]}}
  \\
  \inferrule*[right=Read1]
    {\step{e_1}{e_1'}}
    {\step{e_1[e_2]}{e_1'[e_2]}}
  \and
  \inferrule*[right=Read2]
    {\step{e_2}{e_2'}}
    {\step{v_1[e_2]}{v_1[e_2']}}
  \and
  \inferrule*[right=DoRead]
    { }
    {\step{[v_1,\dots,v_n][i]}{v_i}}
  \and
  \inferrule*[right=Write1]
    {\step{e_1}{e_1'}}
    {\step{e_1[e_2] := e_3}{e_1'[e_2] := e_3}}
  \and
  \inferrule*[right=Write2]
    {\step{e_2}{e_2'}}
    {\step{v_1[e_2] := e_3}{v_1[e_2'] := e_3}}
  \and
  \inferrule*[right=Write3]
    {\step{e_3}{e_3'}}
    {\step{v_1[v_2] := e_3}{v_1[v_2] := e_3'}}
  \and
  \inferrule*[right=DoWrite]
    { }
    {\step{[v_1,\dots,v_n][i] := v}{[v_1,\dots,v_{i-1},v,v_{i+1},\dots,v_n]}}
\end{mathpar}
}

If we take these operational semantics literally, then we are copying
arrays by value all over the place.

\clearpage

\section*{Arrays on the heap}
In hopes of efficiency, we might instead
try to allocate arrays on a heap and pass references around.
A heap is a mapping from addresses to values.

Values no longer include arrays, but instead include \emph{addresses}, written $\ell$.
We extend the operational semantics to pass around heaps everywhere, and update
the rules for arrays as follows.

\vspace{5mm}
{
\def\MathparLineskip {\lineskiplimit=1.2em\lineskip=2.5em plus 0.2em}

{
\setlength{\fboxsep}{5pt}
\fbox{\step{(e,h)}{(e,h)}}
}
\begin{mathpar}
  \inferrule*[right=Alloc]
    {\ell \not\in h}
    {\step{([v_1,\dots,v_n], h)}{(\ell, h \cup \{\ell \mapsto [v_1,\dots,v_n]\})}}
  \and
  \inferrule*[right=DoRead]
    { h(\ell) = [v_1,\dots,v_n] }
    {\step{(\ell[i], h)}{(v_i, h)}}
  \and
  \inferrule*[right=DoWrite]
    { h(\ell) = [v_1,\dots,v_n] \and
      h' = h[\ell\mapsto [v_1,\dots,v_{i-1},v,v_{i+1},\dots,v_n]]
      }
    {\step{(\ell[i] := v, h)}{(\ell, h')}}
\end{mathpar}
}



But \textbf{this changes the meaning of arrays}. For example, the program
\begin{align*}
  &\letin{a}{[0, 0]}\\
  &\letin{b}{a[0] := 1}\\
  &\letin{c}{a[1] := 1}\\
  &b[1] + c[0]
\end{align*}
evaluates to 0 under pure semantics (since the writes to $b$ and $c$
are never observed, so 0 is read in both cases on the last line), but
2 under heap semantics (since the writes to $b$ and $c$ alias and 1 is
read in both cases on the last line).

So we like the heap semantics because operationally it is more
efficient, but adopting it as is would fundamentally change the way
we reason about programs.

We can recover purely functional reasoning even with arrays on the
heap using linear types.

\clearpage
\section*{Linear STLC with arrays on the heap}

{
\noindent
\hskip 1cm plus 1fill minus 0pt
\[
\begin{array}{r@{~}c@{~~}l}
  \tau \bnfeq
       %\unit\\
       %\bnfalt
       \bool\\
       \bnfalt \tyint\\
       \bnfalt \tau\linto\tau\\\\
\end{array}
\]
}

{
\def\MathparLineskip {\lineskiplimit=1.2em\lineskip=2.5em plus 0.2em}

{
\setlength{\fboxsep}{5pt}
\fbox{\haslintype{\gamma}{e}{\tau}}
}
\begin{mathpar}
  \inferrule*[right=Var]
    { }
    {\haslintype{x\Colon\tau}{x}{\tau}}
  \and
  \inferrule*[right=FunBroken]
    {\hastype{\gamma,f\Colon\tau_1\to\tau_2,x\Colon\tau_1}{e}{\tau_2}}
    {\hastype{\gamma}{\fix{f}{x}{\tau_1}{\tau_2}{e}}{\tau_1\linto\tau_2}}
  \and
  \inferrule*[right=App]
    {\haslintype{\gamma_1}{e_1}{\tau_1\to\tau_2} \and
     \haslintype{\gamma_2}{e_2}{\tau_1} \and
     \gamma = \gamma_1\uplus  \gamma_2
     }
    {\haslintype{\gamma}{e_1\ e_2}{\tau_2}}
  \\
  \inferrule*[right=True]
    { }
    {\haslintype{\cdot}{\true}{\bool}}
  \and
  \inferrule*[right=False]
    { }
    {\haslintype{\cdot}{\false}{\bool}}
  \and
  \inferrule*[right=Int]
    { }
    {\haslintype{\cdot}{n}{\tyint}}
  \and
  \inferrule*[right=If]
    {\haslintype{\gamma_1}{e_1}{\bool}\and
     \haslintype{\gamma_2}{e_2}{\tau}\and
     \haslintype{\gamma_3}{e_3}{\tau}\and
     \gamma = \gamma_1 \uplus \gamma_2 \uplus \gamma_3}
    {\haslintype{\gamma}{\IF{e_1}{e_2}{e_3}}{\tau}}
  \and
  \inferrule*[right=Op]
    {\oplus\COLON \tau_1^2 \to \tau_2 \and
     \haslintype{\gamma_1}{e_1}{\tau_1} \and
     \haslintype{\gamma_2}{e_2}{\tau_1} \and
     \gamma = \gamma_1 \uplus \gamma_2
    }
    {\haslintype{\gamma}{e_1\oplus e_2}{\tau_2}}
  \and
  \inferrule*[right=Alloc]
    {\forall i,~\haslintype{\gamma_i}{e_i}{\tau} \and
     \gamma = {\scriptstyle\biguplus_i} \gamma_i
    }
    {\haslintype{\gamma}{[e_1,\dots,e_n]}{\tyarray\ \tau}}
  \and
  \inferrule*[right=Read]
    {\haslintype{\gamma_1}{e_1}{\tyarray\ \tau} \and
     \haslintype{\gamma_2}{e_2}{\tyint} \and
     \gamma = \gamma_1\uplus \gamma_2
    }
    {\haslintype{\gamma}{e_1[e_2]}{\tau}}
  \and
  \inferrule*[right=Write]
    {\haslintype{\gamma_1}{e_1}{\tyarray\ \tau} \and
     \haslintype{\gamma_2}{e_2}{\tyint} \and
     \haslintype{\gamma_3}{e_3}{\tau} \and
     \gamma = \gamma_1 \uplus \gamma_2 \uplus \gamma_3
    }
    {\haslintype{\gamma}{e_1[e_2] := e_3}{\tyarray\ \tau}}

\end{mathpar}
}

The bad example program from before is ruled out by the type
system. \todo{explain why it doesn't typecheck}

In fact, the value semantics and the heap semantics are equivalent for
well-typed programs. The key is that the linear types rule out
aliasing among references.

The linear types also guarantee that all arrays will be deallocated.

\section*{The problem today's reading group paper is trying to solve}

Programming with linear types everywhere by default sucks. It would be nice
to combine linear types with normal types. This paper shows how to do that
in Haskell.

\end{document}
