\documentclass{article}
\usepackage[margin=1in]{geometry}
\usepackage{amsmath}
\usepackage{amsthm}
\usepackage{amssymb}
\usepackage{stmaryrd}
\usepackage{tikz}
\usepackage{mathpartir}
\usepackage{array}
\usepackage{varwidth}

\title{A Missing 505 Lecture: Linear STLC}
\author{Chandrakana Nandi\and James R. Wilcox}
\date{January 26, 2018}

\newcommand{\bnfeq}{&::=&}
\newcommand{\bnfalt}{&~\mid&}


\newcommand{\bool}{\ensuremath{\mathsf{bool}}}
\newcommand{\true}{\ensuremath{\mathsf{true}}}
\newcommand{\false}{\ensuremath{\mathsf{false}}}
\newcommand{\IF}[3]{\ensuremath{\mathsf{if}\ #1\ \mathsf{then}\ #2\ \mathsf{else}\ #3}}

\newcommand{\Colon}{{\hspace{1pt}:\hspace{1pt}}}
\newcommand{\COLON}{\mathrel{\,:\,}}

\newcommand{\LAMBDA}[3]{\lambda #1\Colon#2.\; #3}

\newcommand{\hastype}[3]{\ensuremath{#1\mathrel{\,\vdash\,}#2 \COLON #3}}

\newcommand{\step}[2]{\ensuremath{#1~\longrightarrow~#2}}


\begin{document}
\maketitle

Recall the call-by-value simply typed $\lambda$-calculus with booleans.

{
\noindent
\hskip 1cm plus 1fill minus 0pt
\begin{varwidth}[t]{0.5\textwidth}
\[
\begin{array}{r@{~}c@{~~}l}
  \tau \bnfeq  \bool\\
       \bnfalt \tau\to\tau\\
\end{array}
\]
\end{varwidth}
\hskip 0pt plus 1fill minus 0pt
\begin{varwidth}[t]{0.5\textwidth}
\[
\begin{array}{r@{~}c@{~~}l}
  e \bnfeq     x \\
    \bnfalt    \LAMBDA{x}{\tau}{e}\\
    \bnfalt    e\ e \\
    \bnfalt    \true \\
    \bnfalt    \false \\
    \bnfalt    \IF{e}{e}{e}
\end{array}
\]
\end{varwidth}
\hskip -1cm plus 1fill minus 0pt
\begin{varwidth}[t]{0.5\textwidth}
\[
\begin{array}{r@{~}c@{~~}l}
  v \bnfeq     \LAMBDA{x}{\tau}{e}\\
    \bnfalt    \true \\
    \bnfalt    \false
\end{array}
\]
\end{varwidth}
\hskip 0pt plus 1fill minus 0pt
}

\vspace{1cm}

{
\def\MathparLineskip {\lineskiplimit=1.2em\lineskip=2.5em plus 0.2em}

{
\setlength{\fboxsep}{5pt}
\fbox{\hastype{\Gamma}{e}{\tau}}
}
\begin{mathpar}
  \inferrule*[right=Var]
    {x\Colon\tau\in\Gamma}
    {\hastype{\Gamma}{x}{\tau}}
  \and
  \inferrule*[right=Abs]
    {\hastype{\Gamma,x\Colon\tau_1}{e}{\tau_2}}
    {\hastype{\Gamma}{\LAMBDA{x}{\tau_1}{e}}{\tau_1\to\tau_2}}
  \and
  \inferrule*[right=App]
    {\hastype{\Gamma}{e_1}{\tau_1\to\tau_2} \and
     \hastype{\Gamma}{e_2}{\tau_1}}
    {\hastype{\Gamma}{e_1\ e_2}{\tau_2}}
  \and
  \inferrule*[right=True]
    { }
    {\hastype{\Gamma}{\true}{\bool}}
  \and
  \inferrule*[right=False]
    { }
    {\hastype{\Gamma}{\false}{\bool}}
  \and
  \inferrule*[right=If]
    {\hastype{\Gamma}{e_1}{\bool}\and
     \hastype{\Gamma}{e_2}{\tau}\and
     \hastype{\Gamma}{e_3}{\tau}}
    {\hastype{\Gamma}{\IF{e_1}{e_2}{e_3}}{\tau}}
\end{mathpar}
}

\vspace{1cm}

{
\def\MathparLineskip {\lineskiplimit=1.2em\lineskip=2.5em plus 0.2em}

{
\setlength{\fboxsep}{5pt}
\fbox{\step{e}{e}}
}
\begin{mathpar}
  \inferrule*[right=Beta]
    { }
    {\step{(\LAMBDA{x}{\tau}{e_1})\ v_2}{e_1[v_2/x]}}
  \and
  \inferrule*[right=App1]
    {\step{e_1}{e_1'}}
    {\step{e_1\ e_2}{e_1'\ e_2}}
  \and
  \inferrule*[right=App2]
    {\step{e_2}{e_2'}}
    {\step{v_1\ e_2}{v_1\ e_2'}}
  \and
  \inferrule*[right=IfT]
    { }
    {\step{\IF{\true}{e_2}{e_3}}{e_2}}
  \and
  \inferrule*[right=IfF]
    { }
    {\step{\IF{\false}{e_2}{e_3}}{e_3}}
  \and
  \inferrule*[right=If1]
    {\step{e_1}{e_1'}}
    {\step{\IF{e_1}{e_2}{e_3}}{\IF{e_1'}{e_2}{e_3}}}
\end{mathpar}
}


\end{document}
